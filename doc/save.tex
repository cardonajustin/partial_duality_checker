\documentclass[reprint]{revtex4-2}
\usepackage{graphicx}
\usepackage{amsmath}
\usepackage{amssymb}
\usepackage{physics}


% CITATION CONFIGURATION
%\usepackage[soting=none]{biblatex}
%\addbibresource{references}
\bibliographystyle{plain}
%\let\cite=\supercite
\usepackage{hyperref}

% NOTATION CONFIGURATION
\newtheorem{definition}{Definition}
\newtheorem{proposition}{Proposition}
\newtheorem{theorem}{Theorem}
\newtheorem{lemma}{Lemma}
\newtheorem{corollary}{Corollary}


\begin{document}
\title{Partial Dual Method for Solving QCQPs}
\author{Justin Cardona}
\affiliation{Engineering Physics Department, Polytechnique Montréal}
\begin{abstract}
\end{abstract}
\maketitle


\section{Introduction}
\label{sec:introduction}

\section{Optimization Problem}
\label{sec:optimization}
In this work a numerical scheme is considered for performing inverse design. Firstly the volume where the device must lie is discretized (into a cartesian grid for example). A field in this perspective is represented by a vector with entries that correspond to the polarization current of each cell. For example, if a volume is divided into an $n\times n\times n$ grid then a polarization current can be represented using a vector in $\mathbb{C}^{3n^3}$. Additionally, they can be considered vectors in a Hilbert space with the following inner product:

\begin{align}
	\braket{F}{G} = \int_{\mathbb{R}^3} \dd[3]{x} F^*(x)\cdot G(x)
	\label{eq:inner-product}
\end{align}
Strictly speaking, the Hilbert space being discussed has the above product where the vectors are all elements of $\mathbb{C}^{3n^3}$ that correspond to a physical current/field. In any case, discussing this directly is not particularly important for the rest of this discussion. What is important however, is that physical fields respect the symmetry, linearity, and positive definiteness requirements of a Hilbert space. This can be quickly verified using equation \ref{eq:inner-product}. With this in mind, consider the following quadratic program:

\begin{align}
	\max_{\ket{T}\in\mathbb{C}^{3n^3}} \Im\braket{S}{T} - \ev{O}{T}
\end{align}
In general, $\ket{S}$ can be any element in $\mathbb{C}^{3n^3}$ and $O$ is some general objective linear operator on the space. In the case of Purcell enhancement for example, $O=\text{Asym}\mqty[V^{-\dagger}]$ is an appropriate choice. Without loss of generality, it is convenient in the photonics context to have it represent the source field of the problem (which must be scattered to acheive the desired objective) explicitly. Given a source field $\ket{S}$ and a scattering potential $V$, the total field $\ket{T}$ produced is given by

\begin{align}
	\ket{T} = \qty(V^{-1} - G_0)^{-1}\ket{S},
\end{align}
where $G_0$ is the Green's function of Maxwell's equations for free space. To this objective function constraints are added to enforce power conservation using a hierarchical mean field approach\cite{molesky2020hierarchical}

\section{Padé Algorithm}
\label{sec:algorithm}

\bibliography{references}
% \printbibliography
\end{document}
